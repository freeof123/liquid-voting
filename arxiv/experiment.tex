\section{Experiment}
We compare between our algorithm and traversal algorithm by
recording the maximum consumed gas fee.

\begin{figure}
  \begin{tikzpicture}
\begin{axis}[votegas]
\addplot  plot coordinates {
(10, 139758)
(20, 210619)
(30, 281487)
(40, 352360)
(50, 423240)
(60, 494126)
(70, 565018)
(80, 635916)
(90, 706821)
(100, 777732)
(200, 1487185)
(300, 2197263)
(400, 2907966)
(500, 3619294)
(600, 4331247)
(700, 5043826)
(800, 5757029)
(900, 6470857)
};\addplot  plot coordinates {
(10, 520250)
(20, 580441)
(30, 559707)
(40, 640301)
(50, 619823)
(60, 619695)
(70, 700354)
(80, 700354)
(90, 700418)
(100, 679812)
(200, 739737)
(300, 820587)
(400, 799789)
(500, 799917)
(600, 880448)
(700, 880384)
(800, 859906)
(900, 859906)
(1000, 859970)
(2000, 919895)
(3000, 1000554)
};\legend{traversal,fast}
  \draw [dashed] ( axis cs:0, 6750000) -- ( axis cs:3200, 6750000) node
  [near start, above] {Gas Limit};
\end{axis}
\end{tikzpicture}

  \caption{Voting by delegate chain root.}
  \label{fig:eval:root}
\end{figure}

\begin{figure}
  \begin{tikzpicture}
\begin{axis}[votegas]
\addplot  plot coordinates {
(10, 103692)
(20, 134473)
(30, 165254)
(40, 196036)
(50, 226818)
(60, 257601)
(70, 288384)
(80, 319167)
(90, 349951)
(100, 380735)
(200, 688599)
(300, 996502)
(400, 1304444)
(500, 1612425)
(600, 1920444)
(700, 2228503)
(800, 2536601)
(900, 2844739)
(1000, 3152915)
(2000, 6236825)
};\addplot  plot coordinates {
(10, 536968)
(20, 613237)
(30, 595379)
(40, 689505)
(50, 666551)
(60, 671584)
(70, 760741)
(80, 765774)
(90, 763257)
(100, 742819)
(200, 819088)
(300, 913534)
(400, 895548)
(500, 898193)
(600, 989610)
(700, 992255)
(800, 971817)
(900, 966912)
(1000, 974398)
(2000, 1050794)
(3000, 1144856)
};\legend{traversal,fast}
  \draw [dashed] ( axis cs:0, 6750000) -- ( axis cs:3200, 6750000) node
  [near start, above] {Gas Limit};
\end{axis}
\end{tikzpicture}

  \caption{Voting by delegate chain leaf.}
  \label{fig:eval:tail}
\end{figure}

We conduct the evaluation on Ganache, which is a
personal blockchain for Ethereum development that can be used to deploy contracts, and run tests.
Our implementation can be found here~\footnote{\url{https://github.com/freeof123/liquid-voting/tree/master/ether-eval/contracts}}.
As there is no standard implementation for the traversal algorithm, we
implement it by ourselves.  Specifically, the traversal only happens
when casting a vote, instead of delegating.

Our comparison is from two aspects, \begin{enumerate*}[1) ]%
  \item voting by delegate chain \emph{root}, as illustrated in
  Fig.~\ref{fig:eval:root}, and \item voting by delegate chain
\emph{leaf}, as shown in Fig.~\ref{fig:eval:tail}. \end{enumerate*}.
Notice that the gas limit is about 6,700,000 according to Ganache. Our
evaluation shows that \begin{itemize}
    \item the traversal algorithm performs better when the delegate chain
      is short, like smaller than 100;
    \item our algorithm significantly outperforms the traversal algorithm when
      the delegate chain is long enough;
    \item our algorithm can scale up with very limited gas increasing, while
      the traversal algorithm reaches the gas limit when the delegate chain
      grows up to 1,000.
\end{itemize}.



