\section{Preliminary}
\subsection{Problem Description}
Suppose there are $n$ voters and $m$ candidates. We separate the liquid democracy  problem into two period:
\begin{itemize}
	\item \textbf{Spare period} During spare period, no voting activity is hold. Each voters can arbitrarily delegate, undelegate and change delegate, by sending a massage (transaction) to the blockchain, which is stored on the blockchain. Each voter is allowed to assign at most one delegate. 
	\item \textbf{Voting activity period} Before voting activity period, we suppose that the delegate graph and each voters voting powers are snapshotted, which can be regard as input. We will show in the next section how the voting powers are determined and how to handle the case where there is a loop in the delegate graph and return a no-loop graph. Then after the voting activity begins, each voter can directly vote to a candidate by sending a voting message, with all his delegators' voting powers also cast to that candidate. The on-chain voting status are updated for each voting message. For convenience, we assume that each voter can only vote once during each voting activity, while our algorithm also fit for the case where each voter can change his vote.  
\end{itemize}
The following example illustrates how voting status changes upon receiving voting massages during voting activity period: 
%Suppose there are $n$ voters, each voter has a certain amount of voting power $a[i],i=1,2,...,n$.
%The liquid democracy refers that, each voter (called delegator) can delegate all his voting power to another voter (called delegatee), and his degegatee can further delegate all those voting power to another delegatee. Whenever a delegatee votes to a candidate, by default, all his delegators' (including multi-level) voting powers are also cast to that candidate.
 a (direct) delegate graph $G=(V,E)$ is given, where each note represents a voter, and a direct edge $(u,v)$ represents that voter $v$ delegates to voter $u$. Since by assumption there is no
loop in $G$, thus $G$ is a forest (multiple trees). For convenience, we add a virtual node (indexed 0) that
is pointed by the root of each connected branch. So $G$ is transferred to tree $T$, as
Figure~\ref{fig:1}. That is, there are 12 voters. We further assume that each voter's voting power equals to his index. 

\begin{itemize}
\item At the beginning, nobody
votes. When voter 1 votes for candidate $A$ (as the first voter), $A$ obtains
$1+2+...+12=78$ votes.

\item After voter 1 votes, suppose voter 5 (as the second
voter) votes for candidate $B$. Then $B$ obtains $6+5=11$ votes. $A$'s vote
decreases by 11, turning into 67.

\item Further, voter 3 (as the third voter) votes for candidate $C$, then $C$
obtains $3+4+7+8=22$ votes, $A$'s vote become 45, and $B$'s vote is still 11.
\end{itemize}

\begin{figure*}
  \centering
	\label{fig:1}
	%\includegraphics[width=0.6\textwidth]{1.png}
	%\includegraphics[width=0.6\textwidth]{2.png}
	%
  \begin{tikzpicture}
%\tikzset{grow'=right,level distance=32pt}
%\tikzset{execute at begin node=\strut}
%\tikzset{every tree node/.style={anchor=base west}}
\Tree
[.\node[fsn](n1){1};
  [.\node[sn]{2};
   [.\node[sn]{3};
    [.\node[sn]{4};
     [.\node[fsn](n5){5};
      [.\node[sn]{6};]]]
    [.\node[sn]{7};
     [.\node[sn](n8){8};]]] ]
  [.\node[sn]{9};
   [.\node[sn]{10};]
   [.\node[sn]{11};]
   [.\node[sn]{12};]]
]
\node (a) at ($(n1.east) + (1, 0)$) {A};
\node (b) at ($(n5.west) + (-1, 0)$) {B};
\draw [->, >=stealth', color=blue] (n1) -- (a);
\draw [->, >=stealth', color=blue] (n5) -- (b);

\end{tikzpicture}

	\caption{Tree $T$. We ignore the virtual node with index 0 here.}
\end{figure*}
\textbf{Goal}:
For each voting massage, display the votes of all candidates. (Suppose $m<100$)

\begin{tabular}{|c|c|}
input & output \\
1 A			&		A 78 B 0 C 0
\\
5 B			&		A 67 B 11 C 0
\\
3 C			&		A 45 B 11 C 22
\end{tabular}
\subsection{Blockchain and Smart Contract}
The smart contract of Ethereum supports Turing-complete programming language, which is deployed on the blockchain[]. Users invoke a smart contract by sending a transaction to the smart contract's address, with additional information including the gas fee of the transaction and other incoming parameters, which would further be included in a block. As shown in section 1, the gas fee of a valid transaction are limited by the fixed parameter block\_gas\_limit so that the number of executions of the smart contract are also bounded, that is the so-called "on-chain" time complexity ($O(n)$ is unsatisfactory).  Ethereum clients obtain latest on-chain data through P2P network and implement smart contracts locally through Etheruem virtual machine (EVM). Thus, the space complexity of a smart contract is not issue, which only depends on Ethereum nodes' (clients) local storage. 

It is important to distinguish on-chain smart contract and (open source) cloud computation. The later is still realized by centralized servers, which can not guarantee the codes are correctly executed, while in Ethereum, there is no "center" for executing smart contract: they are executed by every Ethereum nodes, which is reliable as long as the majority of nodes are honest.

\subsection{Merkel Tree}
The Merkel Tree is a common used method for store and verifying on-chain data. One of the key tools is the hash function, $h()$, which is (cryptographically) hard to find collisions and for inverse computation (In Ethereum, SHA3-256 is used). The merkel tree is a full binary tree, where each leaf node of stores the hash value of data to be stored (see Figure x). The value of a intermediate node is the hash value of the combination of its two children. 
\begin{figure*}
	\centering
	\label{fig:merkel}
	\includegraphics[width=1.1\textwidth]{merkel.png}
	%\includegraphics[width=0.6\textwidth]{2.png}
	\caption{Merkel tree, where $h_{A/B/C/D}$ refers to the hash value of data $A/B/C/D$. The black nodes represent the Merkel path of data $A$}
\end{figure*}
The use of Merkel tree is that, the blockchain only need to store the root of the Merkel tree. In order to proof that a data (say data $A$ in the Figure \ref{fig:merkel})belongs to the Merkel tree, $A$ along with its Merkel path (also called Merkel Proof) are required:

\textbf{Merkel path}, which is defined to be a sequence of nodes in the Merkel tree that corresponds to brother of each node on path from  $A$'s leaf node to the root. For example, data $A$'s Merkel path is $(h_B,h_{CD})$. A leaf node together with its correct Merkel path can recover the root of the Merkel tree (called Merkel root). The length of the Merkel path and the time complexity for recovering the Merkel root are all logarithmic to the number of leaf nodes of the Merlek tree.
\subsection{Interval Tree}
Interval tree is also a binary tree, where each node represents a interval and the interval of a parent node is uniformly distributed to its two child nodes, until the interval becomes a singular, to be a leaf node. 
\begin{figure*}
	\centering
	\label{fig:interval}
	%\includegraphics[width=0.6\textwidth]{1.png}
	%\includegraphics[width=0.6\textwidth]{2.png}
	%
	\begin{tikzpicture}[level distance=60pt]
\tikzset{edge from parent/.style=
{draw,
edge from parent path={(\tikzparentnode.south)
-- +(0,-8pt)
-| (\tikzchildnode)}}}
%\tikzset{grow'=right,level distance=32pt}
%\tikzset{execute at begin node=\strut}
%\tikzset{every tree node/.style={anchor=base west}}
\Tree
[.\node[kn](n1){{\color{blue}1}\\$[1,12]$};
  [.\node[kn]{{\color{blue}2}\\$[1,6]$};
   [.\node[kn]{{\color{blue}4}\\$[1,3]$};
    [.\node[kn]{{\color{blue}8}\\$[1,2]$};
     [.\node[kn]{{\color{blue}16}\\$[1,1]$};]
     [.\node[kn]{{\color{blue}17}\\$[2,2]$};]
    ]
    [.\node[kn]{{\color{blue}9}\\$[3,3]$};]
   ]
   [.\node[kn]{{\color{blue}5}\\$[4,6]$};
    [.\node[kn]{{\color{blue}10}\\$[4,5]$};
     [.\node[kn]{{\color{blue}20}\\$[4,4]$};]
     [.\node[kn]{{\color{blue}21}\\$[5,5]$};]
    ]
    [.\node[kn]{{\color{blue}11}\\$[6,6]$};]
   ]
  ]
  [.\node[kn]{{\color{blue}3}\\$[7,12]$};
   [.\node[kn](n6){{\color{blue}6}\\$[7,9]$};]
   [.\node[kn](n7){{\color{blue}7}\\$[10,12]$};]]
]

\node (a) at ($(n6.south) + (0, -0.5)$) {...};
\node (b) at ($(n7.south) + (0, -0.5)$) {...};
\end{tikzpicture}

	\caption{The interval tree of all nodes}
\end{figure*}
See figure \ref{fig:interval} for the interval tree of 12 nodes in Figure \ref{fig:1}. The root are indexed 1 and for node $k$, its child nodes are indexed $2k,2k+1$ respectively. Note that, although some indexes, e.g. 18,19 in Figure \ref{fig:merkel}, does not exist, the space is still used. 

Interval tree usually for maintaining a array where the operations are aiming at a continuous interval. The interval to update begins at the root node, then separates to one or two sub-intervals and recursively executed at the child notes, guaranteeing that the sub-intervals to be update belong to interval of the nodes. The recursion ends when the interval to be update equals to the interval of current nodes. For example, when interval 
[4,8] is to be update, beginning at the root, it separates to [4,6] and [7,8], which are executed at note 2 and 3 respectively. The recursion ends at node 5 and node 12 (which are omitted in Figure \ref{fig:interval}).

Interval tree support find and update operation. For update operation, usually not every leaf node are updated since the update information may stop at an intermediate node, which is recorded as lazy-tag. Find operations need to be  executed recursively starting from the root, and pass down all the lazy-tags until the leaf node is reached. For detial we refer [] to readers.
